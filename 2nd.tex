% @Author: Macsnow
% @Date:   2017-04-24 14:35:48
% @Last Modified by:   Macsnow
% @Last Modified time: 2017-04-24 14:36:16
\begin{enumerate}[1.]
    \item 考虑以下文法$G_1$:
    
        \begin{align*}
            S & \to a | \wedge | (T) \\
            T & \to T,S | S
        \end{align*}
        
        \begin{enumerate}[(1)]
            \item 消去$G_1$的左递归。\sout{然后,对每个非终结符,写出不带回溯的递归子程序。}
            \item 经改写后的文法是否是LL(1)的?给出他的预测分析表。
        \end{enumerate}
        
        \textbf{答:}
        
        \begin{enumerate}[(1)]
            \item
            由于左递归出现在产生式T中,因此引入产生式T'以消除左递归:
            \begin{align*}
                T & \to ST' \\
                T' & \to ,ST' | \epsilon
            \end{align*}
            得到不含左递归的$G_2$:
            \begin{align*}
                S & \to a | \wedge | (T) \\
                T & \to ST' \\
                T' & \to ,ST' | \epsilon
            \end{align*}
            
            \item
            构造$G_2$每个非终结符的的FIRST与FOLLOW:
            \begin{table}[H]
                \centering
                \begin{tabular}{|c|c|c|}
                    \hline
                    $V_N$ & FIRST & FOLLOW \\
                    \hline
                    S & \{$\alpha$, $\wedge$, (\} & \{\#, \textbf{,}, )\} \\
                    \hline
                    T & \{$\alpha$, $\wedge$, (\} & \{)\} \\
                    \hline
                    T' & \{\textbf{,}, $\epsilon$\} & \{)\} \\
                    \hline
                \end{tabular}
                \caption{$G_2$每个非终结符的FIRST的FOLLOW}
                \label{tab:G_2FF}
            \end{table}
            
            根据表\ref{tab:G_2FF}构造预测分析表。
            \begin{table}[H]
                \centering
                \begin{tabular}{|c|c|c|c|c|c|c|}
                    \hline
                    $V_N$ & $\alpha$ & $\wedge$ & ( & ) & \textbf{,} & $\epsilon$ \\
                    \hline
                    S & $S \to \alpha$ & $S \to \wedge$ & $S \to (T)$ & & & \\
                    \hline
                    T & $T \to ST'$ & $T \to ST'$ & $T \to ST'$ & & & \\
                    \hline
                    T' & $T' \to ,ST'$ & & & & & $T \to \epsilon$ \\
                    \hline
                \end{tabular}
                \caption{$G_2$的预测分析表}
                \label{tab:G_2M}
            \end{table}
            
            分析表\ref{tab:G_2M}发现不含多重入口,易知文法$G_2$是LL(1)的。
            
            
        \end{enumerate}
        
    
    \item 对于下面的文法G:
    
        \begin{align*}
            E & \to TE' \\
            E' & \to + E | \epsilon \\
            T & \to FT' \\
            T' & \to T | \epsilon \\
            F & \to PF' \\
            F' & \to *F' | \epsilon \\
            P & \to (E) | a | b | \wedge \\
        \end{align*}
        
        \begin{enumerate}[(1)]
            \item 计算这个文法的每个非终结符的FIRST和FOLLOW.
            \item 证明这个文法是LL(1)的。
            \item 构造它的预测分析表。
            \item \sout{构造它的递归下降分析程序。}
        \end{enumerate}
        
        \textbf{答:}
        
        \begin{enumerate}[(1)]
            \item
            构造表如下:
            \begin{table}[H]
                \centering
                \begin{tabular}{|c|c|c|}
                    \hline
                    $V_N$ & FIRST & FOLLOW \\
                    \hline
                    E & \{(, a, b, $\wedge$\} & \{\#, )\} \\
                    \hline
                    E' & \{+, $\epsilon$\} & \{\#, )\} \\
                    \hline
                    T & \{(, a, b, $\wedge$\} & \{+, \#, )\} \\
                    \hline
                    T' & \{(, a, b, $\wedge$\, $\epsilon$\} & \{+, \#, )\} \\
                    \hline
                    F & \{(, a, b, $\wedge$\} & \{(, a, b, $\wedge$, +, \#, )\} \\
                    \hline
                    F' & \{*, $\epsilon$\} & \{(, a, b, $\wedge$, +, \#, )\} \\
                    \hline
                    P & \{(, a, b, $\wedge$\} & \{*, (, a, b, $\wedge$, +, \#, )\} \\
                    \hline
                \end{tabular}
                \caption{文法G每个非终结符的FIRST和FOLLOW}
                \label{tab:GFF}
            \end{table}
            
            \item
            分别对文法G中每个具有多候选式产生式做判断:
            \begin{itemize}
                \item $E' \to +E | \epsilon$ 
                
                $FIRST(+E) \cap FIRST(\epsilon) = \{+\} \cap \{\epsilon\} = \emptyset$
                
                $FIRST(+E) \cap FOLLOW(E') = \{+\} \cap \{\#, )\} = \emptyset$
                
                \item $T' \to T | \epsilon$
                
                $FIRST(T) \cap FIRST(\epsilon) = \{(, a, b, \wedge\} \cap \{\epsilon\} = \emptyset$
                
                $FIRST(T) \cap FOLLOW(T') = \{(, a, b, \wedge\} \cap \{+, \#, )\} = \emptyset$
                
                \item $F' \to *F' | \epsilon$
                
                $FIRST(*F') \cap FIRST(\epsilon) = \{*\} \cap \{\epsilon\} = \emptyset$
                
                $FIRST(*F') \cap FOLLOW(F') = \{*\} \cap\{(, a, b, \wedge, +, \#, )\} = \emptyset$
                
                \item $P \to (E) | a | b | \wedge$
                
                $FIRST((E)) \cap FIRST(a) \cap FIRST(b) \cap FIRST(\wedge) = \{(\} \cap \{a\} \cap \{b\} \cap \{\wedge\} = \emptyset$
                
                易知$FIRST((E))、 FIRST(a)、 FIRST(b)、 FIRST(\wedge)$两两不相交。
                
                因此原命题G是LL(1)得证。
                
            \end{itemize}
            
            \item
            \begin{table}[H]
                \centering
                \begin{tabular}{|c|c|c|c|c|c|c|c|c|}
                    \hline
                    $V_N$ & ( & a & b & $\wedge$ & ) & + & * & \# \\
                    \hline
                    E & $E \to TE'$ & $E \to TE'$ & $E \to TE'$ & $E \to TE'$ &  &  &  &  \\
                    \hline
                    E' &  &  &  &  &  & $E' \to +E$ & $E' \to \epsilon$ & $E' \to \epsilon$ \\
                    \hline
                    T & $T \to FT'$ & $T \to FT'$ & $T \to FT'$ & $T \to FT'$ &  &  &  &  \\
                    \hline
                    T' & $T' \to T$ & $T' \to T$ & $T' \to T$ & $T' \to T$ & $T' \to \epsilon$ & $T' \to \epsilon$ &  & $T' \to \epsilon$ \\
                    \hline
                    F & $F \to PF'$ & $F \to PF'$ & $F \to PF'$ & $F \to PF'$ &  &  &  &  \\
                    \hline
                    F' & $F' \to \epsilon$ & $F' \to \epsilon$ & $F' \to \epsilon$ & $F' \to \epsilon$ & $F' \to \epsilon$ & $F' \to \epsilon$ & $F' \to *F'$ & $F' \to \epsilon$ \\
                    \hline
                    P & $P \to (E)$ & $P \to a$ & $P \to b$ & $P \to \wedge$ &  &  &  &  \\
                    \hline
                \end{tabular}
                \caption{文法G的预测分析表}
                \label{tab:GM}
            \end{table}
        \end{enumerate}
        
    
    \item 下面文法中,哪些是LL(1)的,说明理由。
    
        \begin{enumerate}[(1)]
            \item
            \begin{align*}
                S & \to Abc \\
                A & \to a | \epsilon \\
                B & \to b | \epsilon
            \end{align*}
            
            \item
            \begin{align*}
                S & \to Ab \\
                A & \to a | B | \epsilon \\
                B & \to b | \epsilon
            \end{align*}
            
            \item
            \begin{align*}
                S & \to ABBA \\
                A & \to a | \epsilon \\
                B & \to b | \epsilon
            \end{align*}
            
            \item
            \begin{align*}
                S & \to aSe | B \\
                B & \to bBe | C \\
                C & \to cCe | d
            \end{align*}
        \end{enumerate}
        
        \textbf{答:}
        
        \begin{enumerate}[(1)]
            \item 是LL(1)的。
            \begin{table}[H]
                \centering
                \begin{tabular}{|c|c|c|}
                    \hline
                    $V_N$ & FIRST & FOLLOW \\
                    \hline
                    S & \{a\} & \{\#, b\} \\
                    \hline
                    A & \{a, $\epsilon$\} & $\emptyset$ \\
                    \hline
                    B & \{b, $\epsilon$\} & $\emptyset$ \\
                    \hline
                \end{tabular}
                \caption{文法$G_{3(1)}$每个非终结符的FIRST和FOLLOW}
                \label{tab:G3_1FF}
            \end{table}
            
            \textbf{证:}
            
            \begin{itemize}
                \item $A \to a | \epsilon$
                
                $FIRST(a) \cap FIRST(\epsilon) = \{a\} \cap \{\epsilon\} = \emptyset$
                
                $FIRST(a) \cap FOLLOW(A) = \{a\} \cap \emptyset = \emptyset$
                
                \item $B \to b | \epsilon$
                
                $FIRST(b) \cap FIRST(\epsilon) = \{b\} \cap \{\epsilon\} = \emptyset$
                
                $FIRST(b) \cap FOLLOW(B) = \{b\} \cap \emptyset = \emptyset$
            \end{itemize}
            
            因此文法$G_{3(1)}$是LL(1)的。
            
            \item 不是LL(1)的。
            \begin{table}[H]
                \centering
                \begin{tabular}{|c|c|c|}
                    \hline
                    $V_N$ & FIRST & FOLLOW \\
                    \hline
                    S & \{a, b\} & \{\#\} \\
                    \hline
                    A & \{a, b, $\epsilon$\} & \{b\} \\
                    \hline
                    B & \{b, $\epsilon$\} & \{b\} \\
                    \hline
                \end{tabular}
                \caption{文法$G_{3(2)}$每个非终结符的FIRST和FOLLOW}
                \label{tab:G3_2FF}
            \end{table}
            
            \textbf{证:}
            
            $A \to a | B | \epsilon$
                
            $FIRST(a) \cap FIRST(B) \cap FIRST(\epsilon) = \{a\} \cap \{b\} \cap \{\epsilon\} = \emptyset$
            
            易知$FIRST(a)、 FIRST(B)、 FIRST(\epsilon)$两两不相交。
            
            $FIRST(a) \cap FOLLOW(A) = \{a\} \cap \{b\} = \emptyset$
            
            $FIRST(B) \cap FOLLOW(A) = \{b, \epsilon\} \cap \{b\} = \{b\} \neq \emptyset$
            
            因此文法$G_{3(2)}$不是LL(1)的。
            
            \item 不是LL(1)的。
            \begin{table}[H]
                \centering
                \begin{tabular}{|c|c|c|}
                    \hline
                    $V_N$ & FIRST & FOLLOW \\
                    \hline
                    S & \{a\} & \{\#\} \\
                    \hline
                    A & \{a, $\epsilon$\} & \{\#, a, b\} \\
                    \hline
                    B & \{a, $\epsilon$\} & \{\#, a, b\} \\
                    \hline
                \end{tabular}
                \caption{文法$G_{3(3)}$每个非终结符的FIRST和FOLLOW}
                \label{tab:G3_3FF}
            \end{table}
            
            \textbf{证:}
            
            $A \to a | \epsilon$
            
            $FIRST(a) \cap FIRST(\epsilon) = \{a\} \cap \{\epsilon\} = \emptyset$
            
            $FIRST(a) \cap FOLLOW(A) = \{a\} \cap \{\#, a, b\} = \{a\} \neq \emptyset$
            
            因此文法$G_{3(3)}$不是LL(1)的。
            
            \item 是LL(1)的。
            \begin{table}[H]
                \centering
                \begin{tabular}{|c|c|c|}
                    \hline
                    $V_N$ & FIRST & FOLLOW \\
                    \hline
                    S & \{a, b, c, d\} & \{\#, e\} \\
                    \hline
                    B & \{b, c, d\} & \{\#, e\} \\
                    \hline
                    C & \{c, d\} & \{\#, e\} \\
                    \hline
                \end{tabular}
                \caption{文法$G_{3(4)}$每个非终结符的FIRST和FOLLOW}
                \label{tab:G3_4FF}
            \end{table}
            
            \textbf{证:}
            
            \begin{itemize}
                \item $S \to aSe | B$
                
                $FIRST(aSe) \cap FIRST(B) = \{a\} \cap \{b, c, d\} = \emptyset$
                
                \item $B \to bBe | C$
                
                $FIRST(bBe) \cap FIRST(C) = \{b\} \cap \{c, d\} = \emptyset$
                
                \item $C \to cCe | d$
                
                $FIRST(cCe) \cap FIRST(d) = \{c\} \cap \{d\} = \emptyset$
            \end{itemize}
            
            因此文法$G_{3(4)}$是LL(1)的。
        \end{enumerate}
        
        
    
    \item 对于下面文法:
    
        \begin{align*}
            Expr & \to - Expr \\
            Expr & \to (Expr) | Var \ ExprTail \\
            ExprTail & \to - Expr | \epsilon \\
            Var & \to id \ VarTail \\
            VarTail & \to (Expr) | \epsilon
        \end{align*}
        
        \begin{enumerate}[(1)]
            \item 构造LL(1)分析表。
            \item \sout{给出对句子id- -id((id))的分析过程。}
        \end{enumerate}
        
        \textbf{答:}
        
        \begin{enumerate}[(1)]
        
            \item 
            \begin{table}[b]
                \centering
                \begin{tabular}{|c|c|c|}
                    \hline
                    $V_N$ & FIRST & FOLLOW \\
                    \hline
                    Expr & \{-, (, id\} & \{\#, )\} \\
                    \hline
                    ExprTail & \{-, $\epsilon$\} & \{\#, )\} \\
                    \hline
                    Var & \{id\} & \{\ \} \\
                    \hline
                    VarTail & \{(, $\epsilon$\} & \{\ \} \\
                    \hline
                \end{tabular}
                \caption{$G_4每个非终结符的FIRST和FOLLOW$}
                \label{tab:G4FF}
            \end{table}
            
            构造LL(1)分析表:
            
            \begin{table}[H]
                \centering
                \resizebox{\textwidth}{!}{
                    \begin{tabular}{|c|c|c|c|c|c|c|}
                        \hline
                        $V_N$ & - & ( & ) & \ & id & \# \\
                        \hline
                        Expr & $Expr \to - Expr$ & $Expr \to (Expr)$ &  &  & $Expr \to Var \ ExprTail$ &  \\
                        \hline
                        ExprTail & $ExprTail \to - Expr$ &  & $ExprTail \to \epsilon$ &  &  & $ExprTail \to \epsilon$ \\
                        \hline
                        Var &  &  &  &  & $Var \to id \ VarTail$ & \\
                        \hline
                        VarTail &  & $VarTail \to (Expr)$ &  & $VarTail \to \epsilon$ &  &  \\
                        \hline
                    \end{tabular}
                }
                \caption{文法$G_4$的LL(1)分析表}
                \label{tab:G4M}
            \end{table}
        \end{enumerate}
\end{enumerate}